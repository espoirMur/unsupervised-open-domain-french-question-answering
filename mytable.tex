\begin{tabular}{l|l|l|l|l}
Question & Top four contexts & Gold Answer & Predicted Answer \\
De quelle langue provient le nom Babel ? & Le nom du poisson renvoie au recit biblique de la Tour de Babel, qui decrit les evenements qui, selon la theologie chretienne et juive, ont conduit Dieu a introduire des langues differentes dans le monde. & Hebreu & Grec \\
 & Comme  WordNet, BabelNet regroupe les mots de differentes langues par groupes de synonymes appeles Babel synsets. Pour chaque Babel synset, BabelNet fournit des definitions textuelles (appelees gloses) en plusieurs langues, obtenues a partir de WordNet et Wikipedia. &  &  \\
 &  Yu, Xiuying. "Review of De l’un au multiple : Traductions du chinois vers les langues europeennes by Viviane Alleton & Michael Lackner." Babel, volume 48, Issue 4, 2002, . DOI: 10.1075/babel.48.4.13xiu. &  &  \\
 & D'autre part, le nom meme de Babel, nom 	\textbf{hebreu} de Babylone, fut tire de l'akkadien bab-ili(m) signifiant « la Porte du Dieu ». On peut donc penser que le dieu de la Tour de Babel ou de la Maison-Dieu n'est pas le vrai Dieu, mais peut-etre son oppose, c'est-a-dire Mammon, dieu personnifiant la richesse. &  &  \\
Que combat Roger pour délivrer angélique? & Le Roland furieux, decrit comme une œuvre majeure inscrite dans la continuite du cycle carolingien, donne naissance a de nombreuses interpretations. L'Arioste s'est tres probablement inspire de textes greco-romains pour composer son œuvre, bien que cela soit difficile a prouver. Plusieurs essais d'interpretations soutiennent que l'Arioste aurait plagie des œuvres antiques. Ainsi, l'hippogriffe est l'equivalent de Pegase et Angelique, de la princesse Andromede, qui est delivree par le heros Persee, egalement en terrassant 	\textbf{un monstre marin} (ce qui rappelle tres fortement la scene, abondamment representee dans l'art, de la delivrance d'Angelique par Roger). & un monstre marin & l’hippogriffe \\
 & Angelique Collet doit delivrer la reine Angelique Limoges retenue prisonniere avec ses gardiens par l'empereur Leviath et ses chevaliers ; elle est aidee par un nouveau personnage, le mysterieux Arios. &  &  \\
 & L'historien des religions Salomon Reinach affirme que Pegase n'a jamais servi de monture a Persee, les multiples representations de la delivrance d'Andromede dans les tableaux des peintres de la Renaissance seraient des erreurs, ainsi que l'etablit une etude de Rensselaer W. Lee, selon laquelle cette figure resulte d'une combinaison du theme de Roger chevauchant l’hippogriffe et terrassant 	\textbf{un monstre marin} pour delivrer Angelique dans le Roland furieux, du mythe de Bellerophon et du mythe de Persee. &  &  \\
 & Il existe toutefois des interpretations du mythe de Pegase, posterieures aux textes mythologiques, ou Athena dompte Pegase et en fait don au heros Persee qui s'envole alors en Ethiopie pour secourir Andromede livree a la colere d'	\textbf{un monstre marin}, qui s'appuient sur le qualificatif de  que le pseudo-Hesiode attribue a Persee, ouvrages de vulgarisation et etudes existent pour certifier que Pegase fut bien la monture de Persee. On peut mettre cette legende posterieure ou Persee delivre Andromede sur le dos de Pegase en parallele ainsi que ses representations en parallele avec celle de Saint Georges terrassant le dragon sur un cheval blanc dans la symbolique chretienne, ainsi qu'avec celle de Roger chevauchant l’hippogriffe et terrassant 	\textbf{un monstre marin} pour delivrer Angelique dans le Roland furieux. &  &  \\
De quel couleur est l'ange musicien jouant de la vièle & Les titres modernes des œuvres correspondent a leur description. Ils sont etablis dans l'ensemble de la litterature scientifique en Ange musicien en rouge jouant du luth et Ange musicien en 	\textbf{vert} jouant de la viele. Neanmoins, concernant ce dernier tableau, la plupart des historiens de l'art specialises en peinture identifient incorrectement l'instrument joue par l'ange avec une viele alors qu'il joue de la . Bien qu'il devrait etre Ange musicien en 	\textbf{vert} jouant de la lira da braccio, c'est donc un titre qui s'est impose de facon impropre, aussi bien en francais qu'en anglais. & vert & Rouge \\
 & A cette occasion le retable est demonte et restructure, comme le montrent les marques de sciage des deux panneaux des Anges musiciens indiquant que leurs dimensions ont ete modifiees. De meme les couleurs du fond ont-elles subi de profonds changements : lAnge musicien en 	\textbf{vert} jouant de la viele presentait un arriere-plan ou le 	\textbf{vert} dominait et un paysage aux couleurs 	\textbf{vert}es et bleues, tandis que lAnge musicien en rouge jouant du luth proposait une niche de couleur rougeatre imitant certainement la pierre. Or, les fonds ont ete repeints en gris, dans le but probable d'uniformiser les couleurs. &  &  \\
  & Les deux tableaux sont dates entre 1495 et 1499. Le premier s'intitule Ange musicien en 	\textbf{vert} jouant de la viele ; il est depuis longtemps attribue au peintre italien de la Renaissance Giovanni Ambrogio de Predis, mais la recherche recente montre qu'il pourrait plutot etre du a Francesco Napoletano, l'un des eleves de Leonard de Vinci. Le second, nomme Ange musicien en rouge jouant du luth, est generalement attribue a Ambrogio de Predis. L'influence de Leonard de Vinci est patente dans le traitement de ces sujets classiques. &  &  \\
  & La dalle funeraire a effigies gravees du chanoine Rigaud d'Aurillac, mort en 1347. L'œuvre, en pierre de liais, est enrichie de marbre (tete aux traits effaces). Coiffe de l'aumusse, les pieds reposant sur un epagneul, symbole de fidelite, le defunt est represente dans une architecture du gothique rayonnant, entoure de deux anges thuriferaires et de dix anges musiciens jouant de la trompette, de la cornemuse, du tambour, de la viele ou encore de l'orgue portatif. &  &  \\
 De quelle société Hal B. Wallis fait-il parti ? & Dans les années 1940, Casablanca était une ville paisible sur l'Atlantique jusqu'au jour où Hal B. Wallis, producteur de la Warner Bros., tombe sur la pièce Everybody Comes to Rick's écrite en 1938 par Murray Burnett et Joan Alison. C'est Irene Lee Diamond, qui a pour mission de trouver de nouvelles idées de scénarios pour le studio, qui découvre la pièce lors d'un voyage à New York. Cette pièce, qui n'a pas été produite, est inspirée des voyages en Europe dans les années 1930 de Murray Burnett. Sa femme ayant de la famille un peu partout sur le Vieux Continent, il peut assister à Vienne à la vie des réfugiés face au nazisme. Lors de ce même voyage, en France, Burnett se rend avec des amis dans une boîte de nuit de Cap Ferrat où il découvre un pianiste noir. Il dit alors à sa femme : « Quel cadre idéal pour une pièce ! » & Warner Bros & Warner Bros \\
  & Les cadres de la Warner, soumis au Code Hays, envisagerent pour de bon de supprimer le mot « syphilis » du film ; mais Hal B. Wallis, president de l'Association des Producteurs Americains, tout en militant pour la prudence, repondit a Warner Bros. que . &  &  \\
  & En 1948, Harold Hecht, ancien choregraphe et desormais agent artistique de Lancaster, obtient que ce dernier soit detache de son contrat de 7 ans avec le producteur de la Paramount Hal Wallis pour un film l'an avec Warner Bros. Hecht et Lancaster fondent la societe de production independante Norma Productions (en honneur a Norma Lancaster) afin de produire Les Amants traques.  &  &  \\
  & La Warner Bros. engage  pour ecrire une premiere version du scenario mais Hal B. Wallis, le producteur delegue du film, n'est pas satisfait par ce premier jet. En effet, le script de Leigh met de cote le personnage de Lady Marian car elle n'apparait pas dans la legende et les ballades originales contant les aventures de Robin des bois. Ce genre de details ainsi que le langage « fleuri » utilise par le scenariste incitent Wallis a engager Norman Reilly Raine pour reprendre la suite de Leigh. Raine est ensuite associe a Seton I. Miller, un scenariste sous contrat avec le studio. Les deux scenaristes commencent alors leurs recherches et les preparatifs necessaires a l'ecriture de leur histoire. &  &  \\
 Le combientième canal à bief de partage de France est le canal de Briare ? & En 1642 a été inauguré le canal de Briare, 	\textbf{premier} canal à bief de partage construit en France. Sa construction a été reprise en 1635 par les frères Boutheroue et Jacques Guyon. Son exploitation a montré les difficultés dues au manque d'eau au niveau du bief de partage en été. Hector Boutheroue, un jeune frère des concessionnaires du canal de Briare va intervenir comme expert en hydraulique auprès des États de Languedoc pour donner son avis sur les projets présentés sur ce canal. Riquet visite le canal de Briare à son inauguration. Le problème de l'alimentation au niveau de son point haut, le seuil de Naurouze, est rapidement devenu un enjeu pour assurer la faisabilité du canal. Riquet achète vers 1652 la seigneurie de Bonrepos puis, jusqu'en 1658, tous les droits sur l'eau de la communauté de Revel située à proximité. Cela lui permet de contrôler l'eau qui alimente Revel grâce à la rivière Sor et un canal de dérivation. C'est sa connaissance de l'hydrographie de la région de la montagne Noire qu'il parcourt avec le fontainier de Revel Pierre Campmas qui permet à Riquet de trouver la solution au problème de l'alimentation en eau du canal, en particulier en été, en faisant déboucher un canal de dérivation non pas dans le Sor et l'Agout mais au seuil de Naurouze en le faisant passer par le seuil de Graissens. & premier & premier \\
  & Le canal de Briare permet a la navigation de relier les fleuves de Loire et de Seine. Il est un des plus anciens canaux de France et le 	\textbf{premier} de type « canal a bief de partage », prototype de tous les canaux modernes. &  &  \\
  & Le canal de Briare permet a la navigation de relier les fleuves de Loire et de Seine et est un des plus anciens canaux de France et le 	\textbf{premier} de type canal a bief de partage, prototype de tous les canaux modernes. Avec les  de son parcours et ses , en suivant principalement les vallees du Loing cote Seine et de la Trezee cote Loire, il relie le canal du Loing, depuis le hameau de Buges dans le Loiret, a la Loire et au canal lateral a la Loire a Briare. Le canal est gere par VNF.  &  &  \\
  & Mis en service en 1642, le canal de Briare est un des plus anciens canaux de France et le 	\textbf{premier} de type canal a bief de partage. Avec les 54 km de son parcours et ses 38 ecluses, en suivant principalement la vallee du Loing, il relie le canal du Loing, depuis le hameau de Buges dans le Loiret (non loin de Montargis), a la Loire et au canal lateral a la Loire a Briare. Il permet a la navigation de relier les fleuves de Loire et de Seine. &  &  \\
 Quel est le travail de Jack Ryan ? & En 1989, il se voit proposer le rôle de Jack Ryan pour jouer dans À la poursuite d'Octobre rouge, adaptation du roman de Tom Clancy. Mais préférant le rôle du Commandant Marko Ramius, déjà réservé pour Sean Connery, il décline l'offre. Finalement, le rôle est obtenu par Alec Baldwin. C'est en 1992, après s'être désengagé d'un projet de la Paramount Pictures, qu'il récupère le rôle de l'agent de la CIA, laissé vacant par Alec Baldwin qui préfère jouer sur scène à Broadway. Harrison Ford interprète alors le héros de Tom Clancy dans le diptyque Jeux de guerre (1992) et Danger immédiat (1994) réalisé par Phillip Noyce. Le scénario de Jeux de guerre doit être réadapté pour le nouvel acteur car il faut passer d'un agent de 35 ans à un autre de 50. Ce premier film est l'occasion de mettre plus en avant le personnage de Jack Ryan et sa famille après un rôle secondaire dans Octobre Rouge. Harrison Ford impose un personnage vulnérable, l'opposé d'un héros d'action sans crainte et sans reproche, ce qui correspond à l'analyste de la CIA et au père de famille qu'est le personnage. Le scénario de Danger immédiat est lancé en même temps que celui de Jeux de guerre. Les deux films sont des succès et c'est au milieu du tournage de celui-ci que le deuxième film est confirmé. & analyste de la CIA & analyste de la CIA \\
  & Du cote americain, Jack Ryan, un analyste de la CIA qui connait Ramius, en arrive a la conclusion que celui-ci souhaite reellement passer a l'Ouest. Les militaires americains ne sont pas convaincus, mais un responsable du gouvernement, le conseiller a la securite nationale Jeffrey Pelt donne trois jours a Ryan pour prouver sa theorie. &  &  \\
  & Un ancien analyste de la CIA, Jack Ryan, est desormais professeur a l'Ecole Navale. A Londres ou il fait une conference, il assiste a une tentative d'assassinat de Lord Holmes, cousin de la reine du Royaume-Uni et ministre, par des terroristes irlandais se reclamant d'une branche dissidente de l'IRA. Jack Ryan intervient, s'empare d'une arme et est blesse juste avant de tuer l'un d'entre eux, permettant l'arrestation du frere de ce dernier Sean Miller. Il sauve ainsi Holmes, ce qui lui vaut une medaille, son nom et sa photo sont dans toute la presse et il est amene a temoigner au proces de Sean Miller. &  &  \\
  & Et en tant qu'acteur : il revient aux series televisees pour preter ses traits a l'analyste de la CIA Jack Ryan, le personnage cree par le romancier Tom Clancy dans une serie televisee disponible exclusivement sur Amazon Prime fin , et produite par Michael Bay. Ce projet est un reboot de la franchise, quatre ans apres l'echec critique et commercial de The Ryan Initiative, un long-metrage sorti en 2013 porte par Chris Pine. &  &  \\
\end{tabular}
